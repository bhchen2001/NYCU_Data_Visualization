%%%%%%%%%%%%%%%%%%%%%%%%%%%%%%%%%%%%%%%%%
% Stylish Curriculum Vitae
% LaTeX Template
% Version 1.1 (September 10, 2021)
%
% This template originates from:
% https://www.LaTeXTemplates.com
%
% Authors:
% Stefano (https://www.kindoblue.nl)
% Vel (vel@LaTeXTemplates.com)
%
% License:
% CC BY-NC-SA 4.0 (https://creativecommons.org/licenses/by-nc-sa/4.0/)
%
%%%%%%%%%%%%%%%%%%%%%%%%%%%%%%%%%%%%%%%%%
% !TEX program = xelatex
\documentclass[a4paper, oneside, final, 12pt]{scrartcl} % Paper options using the scrartcl class

\usepackage{fontspec} % for other font
\usepackage{xeCJK} % for chinese font
\usepackage{hyperref} % for hyper web link
\usepackage{multirow} % for tabular table in learning progress
\usepackage{graphicx} % for image insersion
\usepackage[export]{adjustbox} % for image frame
\usepackage{setspace}
\usepackage{array}
% Define typographic struts, as suggested by Claudio Beccari
%   in an article in TeX and TUG News, Vol. 2, 1993.
\usepackage{mathptmx}
\usepackage{scrlayer-scrpage} % Provides headers and footers configuration
\usepackage{titlesec} % Allows creating custom \section's
\usepackage{marvosym} % Allows the use of symbols
\usepackage{tabularx,colortbl} % Advanced table configurations
% \usepackage{ebgaramond} % Use the EB Garamond font
\usepackage{microtype} % To enable letterspacing
\usepackage{pdfpages} % for showing pdf
\usepackage{pdflscape}
\usepackage{enumitem}
\usepackage{subcaption}
\usepackage{listings}   % highlight the python code
\usepackage{xcolor}
\usepackage{multirow}
\usepackage{cite} %Imports biblatex package
\usepackage[ruled,linesnumbered]{algorithm2e}
\newcommand\mycommfont[1]{\normalsize\ttfamily\textcolor{blue}{#1}}
\SetCommentSty{mycommfont}
% \usepackage[backend=bibtex,bibencoding=ascii,style=authoryear,sorting=none]{bibtex}
% \addbibresource{reference.bib}
% setup the margin
\usepackage[top=1cm, bottom=1cm, right=2cm, left=2cm]{geometry}

% set the style of listing code
\definecolor{codegreen}{rgb}{0,0.6,0}
\definecolor{codegray}{rgb}{0.5,0.5,0.5}
\definecolor{codepurple}{rgb}{0.58,0,0.82}
\definecolor{backcolour}{rgb}{0.95,0.95,0.92}

\lstdefinestyle{mystyle}{
    backgroundcolor=\color{backcolour},   
    commentstyle=\color{codegreen},
    keywordstyle=\color{magenta},
    numberstyle=\tiny\color{codegray},
    stringstyle=\color{codepurple},
    basicstyle=\ttfamily\footnotesize,
    breakatwhitespace=true,         
    breaklines=true,                 
    captionpos=b,                    
    keepspaces=true,                 
    numbers=left,                    
    numbersep=5pt,                  
    showspaces=false,                
    showstringspaces=false,
    showtabs=false,                  
    tabsize=2
}

\lstset{style=mystyle}

% set chinese and english font
\setmainfont{Times New Roman}
\setCJKmainfont[AutoFakeBold=true, AutoFakeSlant=true]{標楷體}

\titleformat{\section}{\Large\raggedright\bfseries}{}{0em}{}[\titlerule] % Section formatting
\titleformat{\subsection}{\large\raggedright\bfseries}{}{0em}{}
\titleformat{\subsubsection}{\normalsize\raggedright\bfseries}{}{0em}{}

% \pagestyle{scrheadings} % Print the headers and footers on all pages

% enable bold and slant chinese font
% \xeCJKsetup{AutoFakeBold=true, AutoFakeSlant=true}

% set the space at the front of paragraph
\setlength{\parindent}{2em}

% disable page number
\pagenumbering{gobble}

\newcommand{\gray}{\rowcolor[gray]{.90}} % Custom highlighting for the work experience and education sections
\newcommand{\Tstrut}{\rule{0pt}{2.6ex}}         % = `top' strut
\newcommand{\Bstrut}{\rule[-0.9ex]{0pt}{0pt}}   % = `bottom' strut
\newcommand{\Tstruth}{\rule{0pt}{4ex}}         % = `top' strut for header
\newcommand{\Bstruth}{\rule[-2.5ex]{0pt}{0pt}}   % = `bottom' strut for header

%----------------------------------------------------------------------------------------
%	FOOTER SECTION
%----------------------------------------------------------------------------------------

% \renewcommand{\headfont}{\normalfont\rmfamily\scshape} % Font settings for footer

% \cofoot{
% \fontsize{12.5}{17}\selectfont % Letter spacing and font size

% \textls[150]{123 Broadway {\large\textperiodcentered} City {\large\textperiodcentered} Country 12345}\\ % Your mailing address
% {\Large\Letter} \textls[150]{john@smith.com \ {\Large\Telefon} (000) 111-1111} % Your email address and phone number
% }

%----------------------------------------------------------------------------------------
\begin{document}

%----------------------------------------------------------------------------------------
%	HEADER SECTION
%----------------------------------------------------------------------------------------


\begin{center}
    {\fontsize{18}{30}\textbf{NYCU 2023 Autumn \\ Data Vislualization \\ Final Project Report Team23}} \\
\end{center}

% list two authors information side by side
  \begin{minipage}[t]{0.45\textwidth}
    \begin{center}
      \textbf{Bo-Han Chen (陳柏翰)} \\
      Student ID: 312551074 \\
      bhchen312551074.cs12@nycu.edu.tw
    \end{center}
  \end{minipage}
  \begin{minipage}[t]{0.45\textwidth}
    \begin{center}
      \textbf{Xu Lin (林煦)} \\
      Student ID: 312553027 \\
      f94061042.cs12@nycu.edu.tw
    \end{center}
  \end{minipage}

\section{Abstract}

  In this project, we develop a data vislualization system for
  Taiwan traffic accident data in 2022.
  Our goal is to find out several insights
  related to the cause and trend of traffic accident,
  which can help the experts and government to improve traffic safety.
  With our system, the user can easily find out the temporal and spatial trend of the accident,
  additionally, the driver-contributed cause and vehicle information 
  are also included, which can be used for further analysis.

\section{Motivation}

  Since there are many people injured or killed in traffic accidents every year,
  so the analysis of the traffic accident data 
  is important and necessary to prevent the accident.
  Some of the accidents are caused by the driver's behavior,
  and some are caused by the road condition 
  and the circumstance such as weather condition at the time that the accident happened.
  Therefore, an easy-to-understand data vislualization 
  can help the experts to analyze the cause of the accident
  and promote the right policy to drivers and pedestrians,
  the government can also use the analysis to improve the traffic safety.
  
  \subsection{Questions to Answer}
  
  The following questions are what we want to answer by visualizing the dataset.
  
  \begin{enumerate}
    \item Which time period has the highest accident rate?
    \item Is there any difference between the accident rate in different cities?
    \item Is there any relationship between the 
    driver-contributed causes and the type of vehicle?
    \item Which part of the vehicle is most likely to be hit and fragile during the crash?
  \end{enumerate}

\section{Dataset}

\begingroup
\raggedright

The dataset we choose is the Taiwan road traffic accident statistics 
in 2022 (\emph{111年傷亡道路交通事故資料}), which contains the details of
the accidents that cause death or injury in Taiwan.
The dataset is public and can be accessed from Open Data Website of Taiwan Government
\footnote{\url{https://data.gov.tw/dataset/161199}.}.

\subsection{Data Description}

The dataset contains 845547 records, 
including 4544 A1 level records and 841003 A2 records.
The A1 level means the accident causes death in within 24 hours,
and the A2 level means the accident causes injury or death in more than 24 hours.
Each record contains 51 attributes, including the time, location,
weather, road type, vehicle type, and the number of death/injury.
Some details such as whether the driver is drunk or not, 
which part of the vehicle hit in the accident, 
and the driver-contributed cause of the accident are also included.

\section{Methodology}

Figure \ref{fig: system_overview} shows the overview of our vislualization system.
The system is devided into 6 parts,
including Taiwan traffic accident map, line chart of death / injury trend, 
stream chart of death / injury trend on different weather condition,
heatmap of accident rate, stacked bar chart of driver-contributed cause and vehicle type,
and the crash position distribution of vehicle.

\begin{figure}[htbp]
  \centering
  \includegraphics[width=1\textwidth]{"./Image/system_overview.png"}
  \caption{System Overview}
  \label{fig: system_overview}
\end{figure}

\subsection{Taiwan Traffic Accident Map}

This map shows the death / injury count of each cities in Taiwan.
The user can directly indicate which city has high casualty count
in the selected time period with the ordinal color scheme.
For detailed information, the user can move the mouse to the city
and the tooltip will show the death / injury count,
drunk driver count, and the mobile phone usage count,
which shows in Figure \ref{fig: map_tooltip}.

\begin{figure}[htbp]
  \centering
  \includegraphics[width=0.7\textwidth]{"./Image/map_tooltip.png"}
  \caption{Map Tooltip}
  \label{fig: map_tooltip}
\end{figure}

\subsection{Line Chart of Death / Injury Trend}

This line chart shows the death / injury trend in the selected time period,
which can help the user to find out the temporal trend of the accident.
The chart will also show fine-grained information when the selected time period is small,
For example, if the user selects the time period of 1 month,
the chart will show the trend of each day in the selected month,
if the selected period is larger than 1 month,
the chart will show the weekly trend.
The tooltip shows the time period and death / injury count of each data point,
which shows in Figure \ref{fig: line_tooltip}.

\begin{figure}[htbp]
  \centering
  \includegraphics[width=1.0\textwidth]{"./Image/line_tooltip.png"}
  \caption{Line Chart Tooltip}
  \label{fig: line_tooltip}
\end{figure}

\subsection{Stream Chart of Accident Count on Different Weather Condition}

This stream chart shows the accident count on different weather condition.
User can find out the accident trend on different weather condition.
The weather condition is divided into 3 categories,
including sunny, cloudy, and rainy.
Figure \ref{fig: stream_tooltip} shows the stream chart in our system,
the tooltip shows the time period and accident count on different weather condition.

\begin{figure}[htbp]
  \centering
  \includegraphics[width=1.0\textwidth]{"./Image/stream_tooltip.png"}
  \caption{Stream Chart Tooltip}
  \label{fig: stream_tooltip}
\end{figure}

\subsection{Grid Heatmap of Accident Count}

The heatmap shows the accident count by time of day and day of week.
This helps user to find out the cyclical trend of the accident,
such as comparing the accident count between weekday and weekend,
and the accident count in the morning and afternoon.
Figure \ref{fig: heatmap_tooltip} shows the heatmap in our system,
the tooltip shows the time period and accident count of each cell.

\begin{figure}[htbp]
  \centering
  \includegraphics[width=0.7\textwidth]{"./Image/heatmap_tooltip.png"}
  \caption{Heatmap Tooltip}
  \label{fig: heatmap_tooltip}
\end{figure}

\subsection{Stacked Bar Chart of Driver-Contributed Cause and Vehicle Type}

This stacked bar chart shows the ratio of vehicle type for each driver-contributed cause.
The causes of accidents are listed from top to bottom in 
descending order according to the number of accidents they caused.
User can find out the vehicle type that is 
most likely to be involved in the corresponding cause.
Figure \ref{fig: stacked_tooltip} shows the stacked bar chart in our system,
on this chart, we can find out that the most frequent cause is
"未注意車前狀態", and the tooltip shows that scooter is the most likely vehicle type
to be involved in this cause, which is 50.9\% of the total accident count.

\begin{figure}[htbp]
  \centering
  \includegraphics[width=0.8\textwidth]{"./Image/stacked_tooltip.png"}
  \caption{Stacked Bar Chart Tooltip}
  \label{fig: stacked_tooltip}
\end{figure}

\subsection{Crash Position Distribution of Vehicle}

This chart shows the crash position distribution of vehicle.
In our dataset, the crash position is divided into 8 categories,
including front, rear, left, right, front-left, front-right, rear-left, and rear-right.
User can find out which part of the vehicle is most likely to be hit
and fragile during the crash by the ordinal color scheme.
Figure \ref{fig: crash_tooltip} shows the crash position distribution of vehicle in our system,
the tooltip shows the crash position and the accident count.

\begin{figure}[htbp]
  \centering
  \includegraphics[width=0.3\textwidth]{"./Image/crash_tooltip.png"}
  \caption{Crash Position Distribution of Vehicle Tooltip}
  \label{fig: crash_tooltip}
\end{figure}

\subsection{Interaction}

The user can select the time period and type of accident data (A1, A2, or both)
in the control panel on top of the system (Figure \ref{fig: control_panel}).
Furthermore, city selection is also available in the map,
by clicking the city, system will show the corresponding information.

\begin{figure}[htbp]
  \centering
  \includegraphics[width=0.8\textwidth]{"./Image/panel.png"}
  \caption{Control Panel}
  \label{fig: control_panel}
\end{figure}

\section{Insights}

\subsection{Temporal \& Spatial Trend}

By selecting both A1 and A2 data, we can first find out the city
that has the high fatality and injury count in 2022 is New Taipei City,
TaoYuan City, Taichung City, and Tainan City in Figure \ref{fig: insight_trend_map}.
Each of them has more that 58000 injury count and 160 death count.
Comparing the casualty count between different cities in Taiwan,
we can find out that the count in eastern Taiwan is much lower than the other cities,
which we can infer is because the traffic density is lower in eastern Taiwan.

\begin{figure}[htbp]
  \centering
  \includegraphics[width=0.5\textwidth]{"./Image/insight_trend_map.png"}
  \caption{Accident Trend in Different City}
  \label{fig: insight_trend_map}
\end{figure}

We can further find out the accident trend in eastern / western Taiwan.
By selecting eastern cities such as Taitung, 
we can find out that the peak of the accident count
is in july and August from Figure \ref{fig: taitung_line},
and the accident count on weekend is abnormaly high shown in Figure \ref{fig: taitung_heatmap}.
We infer that the reason is there are many tourists in eastern Taiwan during summer vacation,
and the traffic density will increase during the weekend, 
which causes more accident than other time period.

% two figures side by side
\begin{figure}[htbp]
  \centering
  \begin{subfigure}[b]{0.6\textwidth}
      \includegraphics[width=\textwidth]{"./Image/eastern_line.png"}
      \caption{Accident Trend in Taitung}
      \label{fig: taitung_line}
  \end{subfigure}
  \begin{subfigure}[b]{0.35\textwidth}
      \includegraphics[width=\textwidth]{"./Image/eastern_heatmap.png"}
      \caption{Accident Count in Taitung}
      \label{fig: taitung_heatmap}
  \end{subfigure}
  \caption{Accident Trend in Taitung}
  \label{fig: taitung}
\end{figure}

For western cities such as Hsinchu,
the trend can be found in Figure \ref{fig: hsinchu}.
We can find out that the casualty count becomes 
higher in winter from Figure \ref{fig: hsinchu_line},
this may cause by the early sunset and rainy weather in winter.
From Figure \ref{fig: hsinchu_heatmap}, we can notice that
the accident count is higher in 8:00-12:00 and 16:00-20:00 on weekday,
which is the time period that people go to work and go home,
so the traffic density is higher than other time period,
which causes more accident.
Furthermore, the accident count on Friday in period of 16:00-20:00
is higher than other weekday, we think the reason is that
people may go to other city for travel or go back to their hometown
after work on Friday
, so the traffic density is higher than other weekday.

\begin{figure}[htbp]
  \centering
  \begin{subfigure}[b]{0.6\textwidth}
      \includegraphics[width=\textwidth]{"./Image/western_line.png"}
      \caption{Accident Trend in Hsinchu}
      \label{fig: hsinchu_line}
  \end{subfigure}
  \begin{subfigure}[b]{0.35\textwidth}
      \includegraphics[width=\textwidth]{"./Image/western_heatmap.png"}
      \caption{Accident Count in Hsinchu}
      \label{fig: hsinchu_heatmap}
  \end{subfigure}
  \caption{Accident Trend in Hsinchu}
  \label{fig: hsinchu}
\end{figure}

These observation can help the government in different cities to
improve the traffic safety by employing more traffic police
in the time period or place that has high accident count.
Traffic education can also be promoted to the drivers and pedestrians
to reduce the accident count in the future.

\subsection{Vehicle Type \& Driver-Contributed Cause}

Regarding the proportion of each vehicle type in the 
causes of accidents throughout the entire year's whole data,
we can find out some interesting insights in Figure \ref{fig: insight_cause}.

\begin{figure}[htbp]
  \centering
  \includegraphics[width=0.8\textwidth]{"./Image/insight_stacked.png"}
  \caption{Proportion of Each Vehicle Type in the Causes of Accidents}
  \label{fig: insight_cause}
\end{figure}

\subsubsection{Motorcycles occupy a larger proportion in all major causes of accidents}

Since motorcycles have the highest number of accidents among all vehicle types, 
it is reasonable that they also have a larger proportion in all major causes of accidents.

\subsubsection{Compared to other vehicle types, 
small cars are more likely to have accidents due to 
"improper right turns," while motorcycles still have a higher 
proportion for "improper left turns"}

In the cause of "improper right turns," small cars have a much higher 
percentage of accidents than motorcycles 
(accounting for 65.3\% of the total number of accidents caused by this reason). 
One possible explanation is that small cars have a blind spot when turning right, 
especially for vehicles coming from the right rear, making it easier to 
overlook potential dangers. However, motorcycles have a higher proportion for 
"improper left turns," this may be due to many riders ignoring the "2 stage left turn" 
signals that are designated for motorcycles at most intersections, 
resulting in a high number of accidents. From this analysis, 
it is important to emphasize that small car drivers need to be 
particularly aware of pedestrians and vehicles on the right, 
especially when turning right, to prevent collisions due to blind spots. 
It is also crucial for motorcycle riders to be more aware and compliant 
with the "2 stage left turn" rules.

\subsubsection{Small trucks have a higher proportion of accidents due to
“illegal parking or improper temporary stopping” compared to other vehicle types}

There are two main reasons for this: first, 
small trucks often need to load and unload goods on streets without dedicated areas, 
forcing drivers to park or stop in inappropriate places; second, 
the larger size of small trucks compared to other vehicles means that 
even temporary improper parking can cause more significant traffic blockages or 
visibility issues, increasing the risk of accidents. Understanding why 
small trucks have a higher proportion of accidents due to 
"illegal parking or improper stopping" can help develop targeted traffic 
management strategies and driver training to reduce the occurrence of such accidents. 
Measures might include providing more parking spaces for trucks, enhancing driver training,
 and optimizing delivery schedules.

\subsection{Crash Position Distribution of Vehicle}

Figure \ref{fig: insight_crash} shows the crash position distribution of vehicle
with all data in 2022.

In the dataset, the concentration of impact locations on the front and sides 
of vehicles suggests a correlation with the causes of accidents. 
For instance, frontal collisions, which are most prevalent 
(indicated by the deep red color in the chart), are likely related to 
"not paying attention to the condition of the car in front." 
This could be due to drivers being distracted, leading to rear-end collisions.

Side impacts, on the other hand, 
could be associated with causes such as "improper left or right turns," 
"improper lane changes or directional changes," or "failing to yield as required." 
These types of accidents might occur when a vehicle is turning and either 
misjudges the distance to an approaching vehicle or does not notice it. 
Improper lane changes can lead to side collisions if drivers do not properly 
check their blind spots or misjudge the speed and distance of vehicles in adjacent lanes.

Additionally, the distribution pattern of impacts could be influenced 
by traffic flow and road design. Intersections, for instance, 
are common sites for both frontal and side impacts due to the crossing 
paths of vehicles and the complexity of driving maneuvers involved.

Understanding the relationship between impact locations and 
accident causes can be crucial for developing targeted safety 
measures and driver education programs. For example, 
enhancing awareness about maintaining safe distances, 
proper lane-changing techniques, and the importance of attentiveness 
at intersections can help reduce the occurrence of both frontal and side collisions.

\begin{figure}[htbp]
  \centering
  \includegraphics[width=0.3\textwidth]{"./Image/insight_position.png"}
  \caption{Crash Position Distribution of Vehicle in 2022}
  \label{fig: insight_crash}
\end{figure}

\section{Challenge and Future Work}

During the preprocessing of the dataset and the selection of features, 
many potentially valuable pieces of information were discarded due to missing data, 
imbalanced distributions, or the inability to effectively compare and 
integrate certain features with others. Additionally, 
the multifaceted nature of traffic accidents, 
influenced by factors like temperature, holidays, road conditions, 
and human behavior, makes it challenging to definitively identify specific 
causes or predict patterns based solely on our dataset.

This difficulty in data analysis highlights the complexity of traffic safety as a subject. 
Traffic incidents are not isolated events but are the result of a 
confluence of various factors, some of which may be outside the scope of the available data. 
For future work, it would be beneficial to integrate additional datasets 
that include more comprehensive environmental, temporal, and behavioral factors. 
This could provide a more holistic view of the circumstances leading to 
traffic accidents and aid in developing more effective prevention strategies.

Moreover, the challenge of dealing with incomplete or unbalanced data underscores 
the need for more robust data collection methods in the field of traffic safety. 
Improving data quality and ensuring a more representative sample would significantly 
enhance the reliability of the analysis. It also opens up possibilities for 
applying advanced data analysis techniques, such as machine learning, 
to uncover deeper insights and more accurately predict accident risks under various conditions.

\section{Conclusion}

The comprehensive analysis of Taiwan's 2022 traffic accident data has illuminated 
the intricate and multifaceted nature of traffic incidents, 
revealing how they are influenced by regional dynamics, temporal trends, 
and specific vehicle-related factors. This study has provided valuable insights, 
addressing the questions raised in our objectives. 
We observed notably higher accident rates in major urban areas compared to eastern Taiwan, 
discerned seasonal variations in accident occurrences, 
and identified a disproportionate involvement of motorcycles and small cars in 
certain accident types. These findings answer the questions initially posed, 
offering a clearer understanding of the variables affecting road safety. 
The insights derived from this study are invaluable for the development of 
targeted traffic management strategies and the formulation of effective driver 
education programs. 

% \subsection{Color Selection}

% For cartography color selection, 
% we use the colorbrewer2\footnote{\url{https://colorbrewer2.org/}.}
% to select the color scheme for our map.
% For ordered data, we choose the viridis color scheme,
% which is colorblind-friendly and perceptually uniform.

\endgroup

% \bibliographystyle{unsrt} % We choose the "plain" reference style
% \bibliography{reference} % Entries are in the "references.bib" file

\end{document}